\documentclass{article}

\title{Homework 1}
\author{Carl Eastlund}
\date{Due \textbf{Wed., Jan. 16} at \textbf{9:00pm}.}

%% Fonts
\usepackage{newcent}
\usepackage{fouriernc}
\renewcommand{\ttdefault}{cmtt}

%% Margins
\usepackage[top=1in,bottom=1in,left=1in,right=1in]{geometry}

%% URL formatting and PDF hyperlinks
\usepackage{url}
\usepackage{hyperref}

\begin{document}

\maketitle

\newcommand\link[2][http://]{\href{#1#2}{\nolinkurl{#2}}}
\newcommand\http[1]{\link[http://]{#1}}
\newcommand\https[1]{\link[https://]{#1}}
\newcommand\email[1]{\link[mailto:]{#1}}

\begin{quotation}

  \noindent \textbf{Collaboration Policy:} Your work on this assignment must be
  your own.  You \emph{may not} copy files from other students in this class,
  from people outside of the class, from the internet, or from any other source.
  You \emph{may not} share files with other students in this class.

  \medskip

  \noindent You \emph{may} discuss the problems, concepts, and general
  techniques used in this assignment with other students, so long as you do not
  share actual solutions.

  \medskip

  \noindent If you are in doubt about what you \emph{may} and \emph{may not} do,
  ask the course instructor before proceeding.  If you violate the collaboration
  policy, you will receive a zero as your grade for this entire assignment and
  you will be reported to OSCCR (\link{northeastern.edu/osccr}).

\end{quotation}

\begin{enumerate}

\pagebreak[1]
\item
  You need to sign up for a Github account in order to submit homework in this
  class.
  \begin{enumerate}
  \item
    Make an account on \link{github.com}.
  \item
    Send an email to the instructor at \email{cce@ccs.neu.edu} with the subject
    ``CS4800 Github Account''.  The body of the email must contain your name and
    your Github account name.  You will receive a response within 24 hours with
    the URL to a private Github account for your work in this course.
  \item
    Your private repository will start with only some of the files you need for
    this assignment.  The public repository at
    \link{github.com/neu-cs4800s13/public} has the files you need.  Use
    \texttt{git merge}, \texttt{git rebase}, or some similar tool to import the
    revision history from the public repository into your private repository.
    You will \emph{not} get full credit for just uploading new copies of each
    file into your repository.
  \item
    Complete the other problems as instructed in your private repository.
    Whatever you submit to your private repository via \texttt{git push} by the
    due date will be graded.
  \end{enumerate}

\pagebreak[1]
\item
  Programming in this course will be done in Racket, and will use additional
  Racket libraries provided for the course.  Write your solutions for this
  exercise in Racket.  Put them in a file named \texttt{solution.rkt} in the
  same directory as this PDF.
  \begin{enumerate}
  \item
    Import the course software from \texttt{software/cs4800.rkt} into your
    solution using
    \href{http://docs.racket-lang.org/reference/require.html}{\texttt{require}}.
  \item
    Write a function \texttt{flatten-lists} to ``flatten'' a nested list
    structure into a single list.  For example, the following expressions:
    \begin{verbatim}
(flatten-lists (list))
(flatten-lists (list 2 3))
(flatten-lists (list 1 (list 2 3) 4))
(flatten-lists (list (list 1 (list 2 3) 4) (list 5 (list 6 7) 8))) \end{verbatim}
    \dots should produce, respectively:
    \begin{verbatim}
(list)
(list 2 3)
(list 1 2 3 4)
(list 1 2 3 4 5 6 7 8) \end{verbatim}
    You may not use any built-in or library functions other than
    \texttt{empty?}, \texttt{cons?}, \texttt{cons}, \texttt{first},
    and \texttt{rest}, or their near-synonyms \texttt{null?}, \texttt{pair?}, 
    \texttt{car}, and \texttt{cdr} (note that \texttt{cons} is always
    \texttt{cons}).
  \item
    For full credit, \texttt{flatten-lists} must run in \(O(n)\) time, where
    \(n\) is the combined length of the lists in its input.

    Use \texttt{define/cost} to define \texttt{flatten-lists} and any helper
    functions.  Define a \texttt{cost-model} for the built-in functions that you
    use.  Use the function \texttt{O?} we defined in
    \texttt{lectures/lecture-2013-01-10.rkt} to demonstrate that
    \texttt{flatten-lists} runs in \(O(n)\) time.

    \textbf{Note:} The \texttt{O?} function cannot guarantee that your function
    runs in \(O(n)\) time.  Getting credit for using \texttt{O?} correctly does
    not guarantee credit for writing \texttt{flatten-lists} correctly, and vice
    versa.
  \item
    For extra credit, write \texttt{flatten-lists} so that every \texttt{cons}
    it creates is part of its final output.  In other words, make sure it never
    copies a \texttt{cons} unnecessarily.
  \end{enumerate}

\pagebreak[1]
\item
  Prose and math in this course will be done in \LaTeX{}.  Write your solutions
  for this exercise in \LaTeX{}.  Put them in a file named \texttt{solution.tex}
  in the same directory as this PDF.  Also submit a rendered PDF of your
  solution named \texttt{solution.pdf} in the same directory as this PDF.

  \begin{enumerate}

  \item Demonstrate that
    \(f(n) = 2n^2 - 3n + 4 \in O(n^2)\).
    That is, choose a specific \(c>0\) and a specific \(n_0 > 0\).
    Then, show that for any possible \(n > n_0\), it must be true that
    \(f(n) \leq c n^2\).

  \item Demonstrate that
    \(f(n) = 3 \sqrt{n} \in O(n)\).
    That is, choose a specific \(c>0\) and a specific \(n_0 > 0\).
    Then, show that for any possible \(n > n_0\), it must be true that
    \(f(n) \leq c n\).

  \item Demonstrate that
    \(f(n) = \Sigma_{i=1}^{n} (2i^2+3i+5) \in O(n^3)\).
    That is, choose a specific \(c>0\) and a specific \(n_0 > 0\).
    Then, show that for any possible \(n > n_0\), it must be true that
    \(f(n) \leq c n^3\).

  \item Demonstrate that
    \(f(n) = \sqrt{n} \not\in O(\log_2 n)\).
    That is, for any possible \(c>0\) and for any possible \(n_0 > 0\), give a
    formula for some \(n > n_0\) and show that \(n\) always satisfies
    \(f(n) > c \log_2 n\).

  \end{enumerate}

\end{enumerate}

\end{document}

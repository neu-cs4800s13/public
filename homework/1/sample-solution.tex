\documentclass{article}

\title{Homework 1}
\author{Sample Solution}
\date{\today}

%% Fonts
\usepackage{newcent}
\usepackage{fouriernc}
\renewcommand{\ttdefault}{cmtt}

%% Margins
\usepackage[top=1in,bottom=1in,left=1in,right=1in]{geometry}

%% URL formatting and PDF hyperlinks
\usepackage{url}
\usepackage{hyperref}

%% Math formatting
\usepackage{amsmath}

\begin{document}

\maketitle

\newcommand\link[2][http://]{\href{#1#2}{\nolinkurl{#2}}}
\newcommand\http[1]{\link[http://]{#1}}
\newcommand\https[1]{\link[https://]{#1}}
\newcommand\email[1]{\link[mailto:]{#1}}

\begin{enumerate}

\pagebreak[1]
\item See repository on \link{github.com}.

\pagebreak[1]
\item See code in \texttt{sample-solution.rkt}.

\pagebreak[1]
\item

  \begin{enumerate}

    \pagebreak[1]
  \item Demonstrate that
    \(f(n) = 2n^2 - 3n + 4 \in O(n^2)\).
    That is, choose a specific \(c>0\) and a specific \(n_0 > 0\).
    Then, show that for any possible \(n > n_0\), it must be true that
    \(f(n) \leq c n^2\).

    \paragraph{Solution:} Solving for \(c\) and \(n_0\) as we go.
    \[
    \begin{array}{rcll}
      2n^2 - 3n + 4 &\leq& cn^2 \\
      2n^2 - 3n + 4n &\leq& cn^2
      & $because $ 4 \leq 4n $ if we choose $ n_0 \geq 1 \\
      2n^2 + n &\leq& cn^2
      & $by simplification$ \\
      2n^2 + n^2 &\leq& cn^2
      & $because $ n \leq n^2 $ given $ n_0 \geq 1 \\
      3n^2 &\leq& cn^2
      & $which is true if we choose $ c \geq 3
    \end{array}
    \]
    Therefore, the equation holds for \(c=3\) and \(n_0=1\).

    \bigskip

    \pagebreak[1]
  \item Demonstrate that
    \(f(n) = 3 \sqrt{n} \in O(n)\).
    That is, choose a specific \(c>0\) and a specific \(n_0 > 0\).
    Then, show that for any possible \(n > n_0\), it must be true that
    \(f(n) \leq c n\).

    \paragraph{Solution:} Solving for \(c\) and \(n_0\) as we go.
    \[
    \begin{array}{rcll}
      3 \sqrt{n} &\leq& cn \\
      3n &\leq& cn
      & $because $ \sqrt{n} \leq n $ if we choose $ n_0 \geq 1 \\
      &&& $and this is true if we choose $ c \geq 3
    \end{array}
    \]
    Therefore, the equation holds for \(c=3\) and \(n_0=1\).

    \bigskip

    \pagebreak[1]
  \item Demonstrate that
    \(f(n) = \Sigma_{i=1}^{n} (2i^2+3i+5) \in O(n^3)\).
    That is, choose a specific \(c>0\) and a specific \(n_0 > 0\).
    Then, show that for any possible \(n > n_0\), it must be true that
    \(f(n) \leq c n^3\).

    \paragraph{Solution:} Solving for \(c\) and \(n_0\) as we go.
    \[
    \begin{array}{rcll}
      \Sigma_{i=1}^{n} (2i^2+3i+5) &\leq& cn^3 \smallskip \\
      2(\Sigma_{i=1}^{n} i^2) + 3(\Sigma_{i=1}^{n} i) + 5 &\leq& cn^3
      & $by linearity of $ \Sigma $ (see Appendix A in the text)$ \smallskip \\
      2(\Sigma_{i=1}^{n} i^2) + 3(\frac{n^2+n}{2}) + 5 &\leq& cn^3
      & $by arithmetic series (see Appendix A again)$ \smallskip \\
      2(\frac{2n^3+3n^2+n}{6}) + 3(\frac{n^2+n}{2}) + 5 &\leq& cn^3
      & $by sum of squares (see Appendix A yet again)$ \smallskip \\
      \frac{2}{3}n^3 + \frac{5}{2}n^2 + \frac{11}{6}n &\leq& cn^3
      & $by simplification$ \smallskip \\
      \frac{2}{3}n^3 + \frac{5}{2}n^3 + \frac{11}{6}n^3 &\leq& cn^3
      & $because $ n \leq n^2 \leq n^3 $ if we choose $ n_0 \geq 1 \smallskip \\
      5n^3 &\leq& cn^3
      & $by simplification; true if we choose $ c \geq 5
    \end{array}
    \]
    Therefore, the equation holds for \(c=5\) and \(n_0=1\).

    \bigskip

    \pagebreak[3]
  \item Demonstrate that
    \(f(n) = \sqrt{n} \not\in O(\log_2 n)\).
    That is, for any possible \(c>0\) and for any possible \(n_0 > 0\), give a
    formula for some \(n > n_0\) and show that \(n\) always satisfies
    \(f(n) > c \log_2 n\).

    \paragraph{Solution:} Solving for \(n\) as we go.
    \[
    \begin{array}{rcll}
      \sqrt{n} &>& c \log_2 n \\
      \sqrt{n} &>& 4c \log_2 \sqrt[4]{n}
      & $by properties of $ \log \\
      \sqrt{n} &>& 4c \sqrt[4]{n}
      & $because $  \sqrt[4]{n} > \log_2 \sqrt[4]{n}
      $ if we choose $ \sqrt[4]{n} > 2 $, meaning $ n > 16 \\
      \sqrt[4]{n} &>& 4c
      & $divide by $ \sqrt[4]n \\
      n &>& 256c
      & $raise both sides to the power of 4$ \\
    \end{array}
    \]
    Therefore, the equation holds for \(n=256c+n_0+16\).

  \end{enumerate}

\end{enumerate}

\end{document}

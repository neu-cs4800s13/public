\documentclass{article}

\title{Homework 3}
\author{Carl Eastlund}
\date{Due \textbf{Mon., Mar. 25} at \textbf{9:00pm}.}

%% Fonts
\usepackage{newcent}
\usepackage{fouriernc}
\renewcommand{\ttdefault}{cmtt}

%% Margins
\usepackage[top=1in,bottom=1in,left=1in,right=1in]{geometry}

%% Math formatting
\usepackage{nicefrac}

%% URL formatting and PDF hyperlinks
\usepackage{url}
\usepackage{hyperref}

\begin{document}

\maketitle

\newcommand\link[2][http://]{\href{#1#2}{\nolinkurl{#2}}}
\newcommand\http[1]{\link[http://]{#1}}
\newcommand\https[1]{\link[https://]{#1}}
\newcommand\email[1]{\link[mailto:]{#1}}

\begin{quotation}

  \noindent \textbf{Collaboration Policy:} Your work on this assignment must be
  your own.  You \emph{may not} copy files from other students in this class,
  from people outside of the class, from the internet, or from any other source.
  You \emph{may not} share files with other students in this class.

  \medskip

  \noindent You \emph{may} discuss the problems, concepts, and general
  techniques used in this assignment with other students, so long as you do not
  share actual solutions.

  \medskip

  \noindent If you are in doubt about what you \emph{may} and \emph{may not} do,
  ask the course instructor before proceeding.  If you violate the collaboration
  policy, you will receive a zero as your grade for this entire assignment and
  you will be reported to OSCCR (\link{northeastern.edu/osccr}).

\end{quotation}

\bigskip

\newcommand\file\texttt
\newcommand\code\texttt

\noindent
As usual, provide the code for your solution in \file{solution.rkt} and provide
mathematical arguments in \file{solution.tex} and \file{solution.pdf}.

\bigskip

\noindent
This assignment deals with strings and characters.  For programming with strings
in Racket, you may find the following functions useful:
\begin{itemize}\setlength\itemsep{0pt}

\item \code{(string-length str)}, which produces the length of the string
  \code{str}.

\item \code{(string-ref str i)}, which produces the \code{i}th character of
  \code{str}, indexed from 0.

\item \code{(string->list str)}, which produces a list of the characters
  contained in the string \code{str}.

\item \code{(list->string chars)}, which produces a string from a list of
  characters \code{chars}.

\item \code{(char=?~a b)}, which reports whether characters \code{a} and
  \code{b} are the same.

\item \code{(char->integer c)}, which produces the Unicode code-point number
  corresponding to the character \code{c}.

\item \code{(integer->char i)}, which produces the character corresponding to
  the Unicode code-point number \code{i}.

\end{itemize}

\noindent\rule{\linewidth}{1pt}

\medskip

\noindent\textbf{Update:} You only need to do \textbf{2 out of 3} of the
problems on this homework.  In addition, each problem has been simplified from
the original posted version.

\begin{enumerate}

  \newpage
  \item

    \noindent\textbf{Update:} You only need to do \textbf{2 out of 3} of the
    problems on this homework.  In addition, this problem has been simplified to
    require memoization; you do not need to make any arguments about a greedy
    version.

    \rule{\linewidth}{1pt}

    Implement \code{(difference one two)} which reports, for any two strings
    \code{one} and \code{two}, the minimum number of operations that can be used
    to transform \code{one} into \code{two}.  The operations to consider are as
    follows:
    \begin{itemize}
    \item \textbf{delete}, which deletes a single character from \code{one}
    \item \textbf{insert}, which inserts a single character into \code{one}
    \item \textbf{replace}, which replaces a single character in \code{one}
    \item \textbf{swap}, which swaps two adjacent characters in \code{one}
    \end{itemize}
    \emph{Note:} Once we insert, replace, or swap a character, we are not
    allowed to change it again by deleting, replacing, or swapping it.  This
    seemingly arbitrary restriction actually simplifies the solution, because it
    means once we process part of a string, we never have to backtrack to
    consider changing it again.

    For example, the difference between \code{"cat"} and \code{"bat"} is 1
    because it takes a single \textbf{replace} operation to replace \code{"c"}
    with \code{"b"}.  The difference between \code{"cab"} and \code{"abc"} is 2
    because we can remove \code{"c"} at the front and insert it again at the
    end.  We cannot use swap twice to achieve the same length---\code{"c"} with
    \code{"a"} and then \code{"c"} with \code{"b"}---because we cannot swap the
    \code{"c"} more than once.  \emph{Note:} The previously
    posted example with \code{"top"} and \code{"pot"} was in error; the
    difference is 2: replace \code{"t"} with \code{"p"} and \code{"p"} with
    \code{"t"}.

    Your solution should traverse the two given strings from left to right,
    considering one operation at a time and advancing along the strings as the
    operations use up one and produce the other.

    \begin{enumerate}

    \item Implement a straightforward recursive solution called
      \code{difference/recursive}.  Analyze its running time.  If its running
      time is exponential or worse, you need only find a lower bound in the
      sense of \(\Omega\).

    \item
      Identify the set of recursive subproblems that \code{difference/recursive}
      must solve.  Find a way to characterize each subproblem as an index into a
      memoization table.  Implement a memoized version of
      \code{difference/recursive} called \code{difference}.  Your code will be
      graded based on the relationship between \code{difference/memoized} and
      \code{difference/recursive}; they should be the same other than the
      changes needed to implement memoization.  Analyze the running time of
      \code{difference}.

    \end{enumerate}

  \newpage
  \item

    \noindent\textbf{Update:} You only need to do \textbf{2 out of 3} of the
    problems on this homework.  In addition, this problem has been simplified to
    require memoization; you do not need to make any arguments about a greedy
    version.

    \rule{\linewidth}{1pt}

    Implement \code{(shared one two)} which reports, for any two strings
    \code{one} and \code{two}, the length of their longest common substring.
    For instance, the longest common substring of \code{"abracadabra"} and
    \code{"bric-a-brac"} is \code{"brac"}, which has length 4.

    Your solution should consider all possible positions for a shared substring
    between the two inputs.

    \begin{enumerate}

    \item Implement a straightforward recursive solution called
      \code{shared/recursive}.  Analyze its running time.  If its running
      time is exponential or worse, you need only find a lower bound in the
      sense of \(\Omega\).

    \item
      Identify the set of recursive subproblems that \code{shared/recursive}
      must solve.  Find a way to characterize each subproblem as an index into a
      memoization table.  Implement a memoized version of
      \code{shared/recursive} called \code{shared}.  Your code will be graded
      based on the relationship between \code{shared/memoized} and
      \code{shared/recursive}; they should be the same other than the changes
      needed to implement memoization.  Analyze the running time of
      \code{shared}.

    \end{enumerate}

  \newpage
  \item

    \noindent\textbf{Update:} You only need to do \textbf{2 out of 3} of the
    problems on this homework.  In addition, this problem has been simplified to
    require a specific greedy choice; you do not need to write a memoized
    version.

    \rule{\linewidth}{1pt}

    Implement \code{(compress str)}, which reports the minimum number of bits
    necessary to represent the characters in the string \code{str} using a
    simple compression system.  In this system, every character is represented
    by a unique sequence of bits.  These sequences are determined by a binary
    tree whose leaves are the unique characters used in the string.  The path to
    each character determines its encoding: the sequence contains a 0 bit for
    every left branch and a 1 bit for every right branch.  The function
    \code{compress} must account for all possible encodings, meaning all
    possible binary trees containing the set of characters in the string.

    For example, the minimum number of bits to encode \code{"tart"} is 6: one
    bit for \code{"t"} and two each for \code{"a"} and \code{"r"}, meaning a
    tree with \code{"t"} on one side and a subtree with \code{"a"} and
    \code{"r"} on the other.  The encodings \code{"t"=0/"a"=10/"r"=11} and
    \code{"t"=1/"a"=00/"r"=01} are examples.

    Your solution should start with a sequence of leaves corresponding to the
    unique characters in the string, and build up a total encoding by repeatedly
    choosing which of the trees so far to combine into a new node.

    \begin{enumerate}

    \item Implement a straightforward recursive solution called
      \code{compress/recursive}.  Analyze its running time.  If its running
      time is exponential or worse, you need only find a lower bound in the
      sense of \(\Omega\).

    \item The \emph{weight} of a given encoding tree is the total number of
      characters in the input string that are represented by that tree.  Argue
      that at each step in the algorithm, combining the two encoding trees of
      least weight ultimately yields the shortest encoding of the string.

    \item Implement a greedy version of \code{compress/recursive} called
      \code{compress}.  Your code will be graded based on the relationship
      between \code{shared/memoized} and \code{shared/recursive}; they should be
      the same other than the changes needed to make a greedy choice.  Analyze
      the running time of \code{shared}.

    \end{enumerate}

\end{enumerate}

\end{document}

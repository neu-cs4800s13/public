\documentclass{article}

\title{Homework 3: Sample Solution}
\author{Carl Eastlund}
\date{}

%% Fonts
\usepackage{newcent}
\usepackage{fouriernc}
\renewcommand{\ttdefault}{cmtt}

%% Margins
\usepackage[top=1in,bottom=1in,left=1in,right=1in]{geometry}

%% Math formatting
\usepackage{nicefrac}

%% URL formatting and PDF hyperlinks
\usepackage{url}
\usepackage{hyperref}

\begin{document}

\maketitle

\newcommand\link[2][http://]{\href{#1#2}{\nolinkurl{#2}}}
\newcommand\http[1]{\link[http://]{#1}}
\newcommand\https[1]{\link[https://]{#1}}
\newcommand\email[1]{\link[mailto:]{#1}}

\bigskip

\newcommand\file\texttt
\newcommand\code\texttt
\newcommand\?{\mbox{\code{?}}}

\begin{enumerate}

\item
  \begin{enumerate}

  \item The running time for \code{difference/recursive} is \(T(n)\), where
    \(n\) is the sum of the lengths of the two input strings.  For
    \code{difference-between/recursive} and its helpers, the length includes
    only characters from indices \code{i} and \code{j} in the respective
    strings.

    The running time for \code{difference-between/recursive} can be stated as
    this recurrence relation:
    \[T(n) = T(n-2) + T(n-1) + T(n-1) + T(n-2) + T(n-4) + 1\]
    Here, the first four recursive references to \(T\) correspond to the
    replace, insert, delete, and swap operations, and the fifth corresponds to
    the case where we ``copy'' a character by performing no operations.  We can
    find a lower bound on \(T(n)\) by reducing all the \(T(n-1)\) and \(T(n-2)\)
    terms with \(T(n-4)\) and combining them.  This gives us:
    \[T(n) = 5T(n-4) + 1\]
    This is in \(Theta(5^{\nicefrac{n}{4}})\), or
    \(\Theta((\sqrt[4]{5})^n)\).

  \item The memoization table for \code{difference} has \((m+1) \times (n+1)\)
    entries, one for every pair of locations in the input strings of lengths
    \(m\) and \(n\) respectively.  It takes \(\Theta(1)\) time to compute the
    solution at each entry in terms of others, so \code{difference} runs in
    \(O(mn)\) time.

  \end{enumerate}

\item
  \begin{enumerate}

  \item The running time for \code{shared/recursive} is \(T(n)\), again using
    \(n\) to mean the sum of the lengths of the two input strings.  Again, we
    consider the indices passed to helper functions to effectively mark the new
    start of the string.

    The running time for \code{shared-at/recursive} is \(O(n)\); it simply
    traverses both strings as far as both have the same characters.

    The running time for \code{max-shared-starting-at/recursive} can be stated
    as this recurrence relation:
    \[T(n) = 2T(n-1) + O(n)\]
    In both cases where it recurs, it drops a single character off of one of its
    inputs.  A lower bound on this recurrence is \(T(n)=2T(n-1)+\Theta(1)\),
    which is in \(\Theta(2^n)\).

  \item The two memoization tables for \code{shared} each have \((m+1) \times
    (n+1)\) entries, one for every pair of locations in the input strings of
    lengths \(m\) and \(n\) respectively.  Each step of both \code{shared-at}
    and \code{max-shared-starting-at} takes \(\Theta(1)\) time to compute a
    solution based on other entries in the table.  The running time of
    \code{shared} is therefore \(O(mn)\).

  \end{enumerate}

\end{enumerate}

\end{document}

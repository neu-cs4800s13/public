\documentclass{article}

\title{Homework 3}
\author{Carl Eastlund}
\date{Due \textbf{Wed., Feb. 27} at \textbf{9:00pm}.}

%% Fonts
\usepackage{newcent}
\usepackage{fouriernc}
\renewcommand{\ttdefault}{cmtt}

%% Margins
\usepackage[top=1in,bottom=1in,left=1in,right=1in]{geometry}

%% Math formatting
\usepackage{nicefrac}

%% URL formatting and PDF hyperlinks
\usepackage{url}
\usepackage{hyperref}

\begin{document}

\maketitle

\newcommand\link[2][http://]{\href{#1#2}{\nolinkurl{#2}}}
\newcommand\http[1]{\link[http://]{#1}}
\newcommand\https[1]{\link[https://]{#1}}
\newcommand\email[1]{\link[mailto:]{#1}}

\begin{quotation}

  \noindent \textbf{Collaboration Policy:} Your work on this assignment must be
  your own.  You \emph{may not} copy files from other students in this class,
  from people outside of the class, from the internet, or from any other source.
  You \emph{may not} share files with other students in this class.

  \medskip

  \noindent You \emph{may} discuss the problems, concepts, and general
  techniques used in this assignment with other students, so long as you do not
  share actual solutions.

  \medskip

  \noindent If you are in doubt about what you \emph{may} and \emph{may not} do,
  ask the course instructor before proceeding.  If you violate the collaboration
  policy, you will receive a zero as your grade for this entire assignment and
  you will be reported to OSCCR (\link{northeastern.edu/osccr}).

\end{quotation}

\bigskip

\newcommand\file\texttt
\newcommand\code\texttt
\newcommand\?{\mbox{\code{?}}}

You will implement five new datatypes in this assignment.  Three are concrete
datatypes; I will tell you what sort of representation to use, and you will
implement that in Racket.  Two are abstract datatypes; I will only tell you what
operations they need to support, and you must both design the representation and
implement your design in Racket.

For each datatype, you will be assigned a set of operations that the datatype
must support, and an upper-bound on the running time of the operations.  In each
case, you will be asked to describe your design and analyze its efficiency in
\LaTeX{} (in \file{solution.tex} and \file{solution.pdf}) and implement it in
Racket (in \file{solution.rkt}).

For the concrete datatypes, the \emph{only} compound data structures you may use
in Racket are lists, arrays, boxes, and \code{struct} definitions.  The
\emph{only} built-in operations you may use on these data structures are those
that run in \(\Theta(1)\) time, plus \code{make-vector}, which runs in
\(\Theta(n)\) time.  You must implement all other operations yourself.  For
reference, all the functions defined by \code{struct} are in \(\Theta(1)\), as
are \code{empty\?}, \code{cons\?}, \code{cons}, \code{first}, \code{rest},
\code{vector-length}, \code{vector-ref}, \code{vector-set!}, \code{box\?},
\code{box}, and \code{unbox}.

For the abstract datatypes, you may use lists, arrays, boxes, and \code{struct}s
as described above; you may also use any of the other datatypes you have
implemented.

\begin{enumerate}

\item Concrete datatypes:
  \begin{enumerate}
  \item Doubly-linked lists.
  \item Balanced binary trees.
  \item Hash tables.
  \end{enumerate}

\item Abstract datatypes:
  \begin{enumerate}
  \item Sets.
  \item Catenable deques.
  \end{enumerate}

\end{enumerate}

\end{document}

\documentclass{article}

\title{Homework 3}
\author{Carl Eastlund}
\date{Due \textbf{Wed., Feb. 27} at \textbf{9:00pm}.}

%% Fonts
\usepackage{newcent}
\usepackage{fouriernc}
\renewcommand{\ttdefault}{cmtt}

%% Margins
\usepackage[top=1in,bottom=1in,left=1in,right=1in]{geometry}

%% Math formatting
\usepackage{nicefrac}

%% URL formatting and PDF hyperlinks
\usepackage{url}
\usepackage{hyperref}

\begin{document}

\maketitle

\newcommand\link[2][http://]{\href{#1#2}{\nolinkurl{#2}}}
\newcommand\http[1]{\link[http://]{#1}}
\newcommand\https[1]{\link[https://]{#1}}
\newcommand\email[1]{\link[mailto:]{#1}}

\begin{quotation}

  \noindent \textbf{Collaboration Policy:} Your work on this assignment must be
  your own.  You \emph{may not} copy files from other students in this class,
  from people outside of the class, from the internet, or from any other source.
  You \emph{may not} share files with other students in this class.

  \medskip

  \noindent You \emph{may} discuss the problems, concepts, and general
  techniques used in this assignment with other students, so long as you do not
  share actual solutions.

  \medskip

  \noindent If you are in doubt about what you \emph{may} and \emph{may not} do,
  ask the course instructor before proceeding.  If you violate the collaboration
  policy, you will receive a zero as your grade for this entire assignment and
  you will be reported to OSCCR (\link{northeastern.edu/osccr}).

\end{quotation}

\bigskip

\newcommand\code[1]{\mbox{\texttt{#1}}}
\newcommand\hastype{\ensuremath{:~}}
\newcommand\type[1]{\ensuremath{\mathit{#1}}}
\newcommand\Bool{\type{Boolean}}
\newcommand\Num{\type{Number}}
\newcommand\Set{\type{Set}}

\begin{enumerate}

\item Implement sets with the following operations:
  \begin{list}{}
    \item \ensuremath{\code{(empty-set)} \hastype \Set{}}
    \item \ensuremath{\code{(in{\code{?}} \Num{} \Set)} \hastype \Bool}
    \item \ensuremath{\code{(insert-into \Num{} \Set)} \hastype \Set}
    \item \ensuremath{\code{(remove-from \Num{} \Set)} \hastype \Set}
  \end{list}

\item Associative map representation.

\item Copyable, extensible sequence representation.

\item Catenable deque representation.

\end{enumerate}

\end{document}

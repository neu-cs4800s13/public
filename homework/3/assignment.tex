\documentclass{article}

\title{Homework 3}
\author{Carl Eastlund}
\date{Due \textbf{Wed., Feb. 27} at \textbf{9:00pm}.}

%% Fonts
\usepackage{newcent}
\usepackage{fouriernc}
\renewcommand{\ttdefault}{cmtt}

%% Margins
\usepackage[top=1in,bottom=1in,left=1in,right=1in]{geometry}

%% Math formatting
\usepackage{nicefrac}

%% URL formatting and PDF hyperlinks
\usepackage{url}
\usepackage{hyperref}

\begin{document}

\maketitle

\newcommand\link[2][http://]{\href{#1#2}{\nolinkurl{#2}}}
\newcommand\http[1]{\link[http://]{#1}}
\newcommand\https[1]{\link[https://]{#1}}
\newcommand\email[1]{\link[mailto:]{#1}}

\begin{quotation}

  \noindent \textbf{Collaboration Policy:} Your work on this assignment must be
  your own.  You \emph{may not} copy files from other students in this class,
  from people outside of the class, from the internet, or from any other source.
  You \emph{may not} share files with other students in this class.

  \medskip

  \noindent You \emph{may} discuss the problems, concepts, and general
  techniques used in this assignment with other students, so long as you do not
  share actual solutions.

  \medskip

  \noindent If you are in doubt about what you \emph{may} and \emph{may not} do,
  ask the course instructor before proceeding.  If you violate the collaboration
  policy, you will receive a zero as your grade for this entire assignment and
  you will be reported to OSCCR (\link{northeastern.edu/osccr}).

\end{quotation}

\bigskip

\newcommand\file\texttt
\newcommand\code\texttt
\newcommand\?{\mbox{\code{?}}}

You will implement three new datatypes in this assignment.  Three are concrete
datatypes; I will tell you what sort of representation to use, and you will
implement that in Racket.  Two are abstract datatypes; I will only tell you what
operations they need to support, and you must both design the representation and
implement your design in Racket.

For each datatype, you will be assigned a set of operations that the datatype
must support, and an upper-bound on the running time of the operations.  In each
case, you will be asked to describe your design and analyze its efficiency in
\LaTeX{} (in \file{solution.tex} and \file{solution.pdf}) and implement it in
Racket (in \file{solution.rkt}).

For implementing the datatypes, the \emph{only} compound data structures you may
use in Racket are lists, arrays, boxes, and \code{struct} definitions.  The
\emph{only} built-in operations you may use on these data structures are those
that run in \(\Theta(1)\) time, plus \code{make-vector}, which runs in
\(\Theta(n)\) time.  You must implement all other operations yourself.  For
reference, all the functions defined by \code{struct} are in \(\Theta(1)\), as
are \code{empty\?}, \code{cons\?}, \code{cons}, \code{first}, \code{rest},
\code{vector-length}, \code{vector-ref}, \code{vector-set!}, \code{box\?},
\code{box}, and \code{unbox}.

In each case, any sequence of \(n\) operations must run in \(O(n \log n)\) time
in the average case.  You can accomplish this using worst-case bounds as with
our ``D'' heaps, or with amortized bounds as in our ``growable sequences'', or
with average-case bounds as in hash tables.

In each case, a data structure containing \(n\) values must use at most
\(\Theta(n)\) space.  Every \code{cons} cell, box, or \code{struct} takes up
\(\Theta(1)\) space plus the space for its contents; every vector of length
\(n\) takes up \(n\) space plus the space for its contents.  Numbers, strings,
symbols, booleans, \code{empty} and \code{(void)} take up \(\Theta(1)\) space
each, for our purposes.  (Technically, numbers and strings can be arbitrarily
large, but we can consider only ``small'' numbers and strings for our purposes.)

\newcommand\ty[1]{\ensuremath{\mathit{#1}}}
\newcommand\isty{\ensuremath{~:~}}
\newcommand\Num{\ty{Number}}
\newcommand\Str{\ty{String}}
\newcommand\Key{\Num}
\newcommand\Val{\Str}
\newcommand\Elem{\Num}
\newcommand\Assoc{\ty{AssocMap}}
\newcommand\Set{\ty{Set}}
\newcommand\Q{\ty{Queue}}
\newcommand\X{\Str}
\newcommand\Bool{\ty{Boolean}}
\newcommand\U{\mbox{ or }}

\begin{enumerate}

\item Implement associative maps.  In most representations, the \code{unassign}
  operation will be the most subtle, so bear it in mind during the design
  phase.
  \begin{list}{}
  \item \(\code{(fresh-assoc)} \isty \Assoc{}\) \\
    Creates a fresh associative map containing no associations.
  \item \(\code{(assign \Key{} \Val{} \Assoc{})} \isty \Assoc{}\) \\
    Adds an association that maps the given \Key{} to the given \Val{},
    overwriting any existing mapping for \Key{},
    returning the resulting \Assoc{}.  Your implementation may construct a new
    \Assoc{} for the result, or mutate and return the original \Assoc{}.
  \item \(\code{(unassign \Key{} \Assoc{})} \isty \Assoc\) \\
    Removes any existing association for \Key{}, returning the resulting
    \Assoc{}.  Your implementation may construct a new \Assoc{} for the result,
    or mutate and return the original \Assoc{}.
  \item \(\code{(lookup \Key{} \Assoc{})} \isty (\Val{} \U \code{\#false})\)
    Looks for an association for \Key{}.  If one exists, returns the
    corresponding \Val{}.  Otherwise, returns \code{\#false}.
  \end{list}
  \begin{enumerate}
  \item What is your data definition for \Assoc{}?  Include all invariants
    necessary to achieve your asymptotic running time and space usage.
  \item Analyze the space used by your representation:
    \begin{enumerate}
    \item Argue that any \Assoc{} matching your data definition that contains
      \(n\) associations uses at most \(O(n)\) space.
    \item Argue that \code{fresh-assoc} produces a valid \Assoc{}.
    \item Argue that \code{assign} produces a valid \Assoc{}.
    \item Argue that \code{unassign} produces a valid \Assoc{}.
    \end{enumerate}
  \item Analyze the running time for each operation:
    \begin{enumerate}
    \item State whether your analysis is worst case, average case, or
      amortized.
    \item Argue that \code{fresh-assoc} takes \(O(1)\) time.
    \item Argue that \code{assign} takes \(O(\log n)\) time for an
      input that contains \(n\) associations.
    \item Argue that \code{unassign} takes \(O(\log n)\) time for an
      input that contains \(n\) associations.
    \item Argue that \code{lookup} takes \(O(\log n)\) time for an
      input that contains \(n\) associations.
    \end{enumerate}
  \end{enumerate}

\item Implement sets.  In most representations, the \code{without} operation
  will be the most subtle, so bear it in mind during the design phase.
  \begin{list}{}
  \item \(\code{(empty-set)} \isty \Set{}\)
  \item \(\code{(in? \Elem{} \Set{})} \isty \Bool{}\)
  \item \(\code{(extend \Elem{} \Set{})} \isty \Set{}\)
  \item \(\code{(without \Elem{} \Elem{} \Set{})} \isty \Set{}\)
  \end{list}

\item Implement double-ended queues.  In most representations, the
  \code{concatenate} operation will be the most subtle, so bear it in mind
  during the design phase.
  \begin{list}{}
  \item \(\code{(new-queue)} \isty \Q{}\)
  \item \(\code{(full\? \Q{})} \isty \Bool{}\)
  \item \(\code{(add-leftmost \X{} \Q{})} \isty \Q{}\)
  \item \(\code{(add-rightmost \X{} \Q{})} \isty \Q{}\)
  \item \(\code{(get-leftmost \Q{})} \isty \X{}\)
  \item \(\code{(get-rightmost \Q{})} \isty \X{}\)
  \item \(\code{(drop-leftmost \Q{})} \isty \Q{}\)
  \item \(\code{(drop-rightmost \Q{})} \isty \Q{}\)
  \item \(\code{(concatenate \Q{} \Q{})} \isty \Q{}\)
  \end{list}

\end{enumerate}

\end{document}

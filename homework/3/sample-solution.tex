\documentclass{article}

\title{Homework 3}
\author{Carl Eastlund}
\date{Due \textbf{Wed., Feb. 27} at \textbf{9:00pm}.}

%% Fonts
\usepackage{newcent}
\usepackage{fouriernc}
\renewcommand{\ttdefault}{cmtt}

%% Margins
\usepackage[top=1in,bottom=1in,left=1in,right=1in]{geometry}

%% Math formatting
\usepackage{nicefrac}

%% URL formatting and PDF hyperlinks
\usepackage{url}
\usepackage{hyperref}

\begin{document}

\maketitle

\newcommand\link[2][http://]{\href{#1#2}{\nolinkurl{#2}}}
\newcommand\http[1]{\link[http://]{#1}}
\newcommand\https[1]{\link[https://]{#1}}
\newcommand\email[1]{\link[mailto:]{#1}}

\bigskip

\newcommand\file\texttt
\newcommand\code\texttt
\newcommand\?{\mbox{\code{?}}}
\newcommand\ty\texttt

\begin{enumerate}

\item Associative Maps.
  \begin{enumerate}
  \item An \ty{AssocMap} is a \ty{Tree}:

\begin{verbatim}
;; A Tree is one of:
;; - empty
;; - (node Integer Key Value Tree Tree)
(struct node [height key value left right] #:transparent)
;; where:
;; - for any node N in left, left.key < key
;; - for any node N in right, key < right.key
;; - height = 1 + max(left.height, right.height)
;; - left.height <= right.height + 1
;; - right.height <= left.height + 1

;; A Key is a Number
;; A Value is a String
\end{verbatim}

  \textbf{Note:} Because the left and right subtrees of a node must have heights
  within 1 of each other, the maximum height of a tree is \(O(\log n)\).  The
  argument is based on the fact that we can construct the smallest possible tree
  for any height \(n\).  For height 0, the tree is \code{empty}.  For height 1,
  the tree is a singleton node.  For any higher height \(n\), the smallest
  possible tree is a node with the smallest tree of height \(n-1\) on the left
  and the smallest tree of height \(n-1\) on the right.  Adding the solutions
  for \(n-1\) and \(n-2\) suggest the Fibonacci numbers; we can verify that the
  smallest tree of height \(n\) has a number of leaves equal to one less than
  the \(n+1\)st Fibonacci number.  Since the Fibonacci numbers grow
  exponentially, the minimum number of nodes for a given height therefore grows
  exponentially.  The maximum height for a given number of nodes is the inverse
  of this function, and it therefore grows logarithmically.  Interestingly, we
  get a logarithmic \emph{height} bound without guaranteeing any constant factor
  bound for the relative \emph{size} of the left and right subtrees.
  
  \item
    \begin{enumerate}
    \item Every key/value pair occupies a node, which takes \(\Theta(1)\)
      space.  Because keys are strictly increasing from left to right,
      there can be no duplicate keys, so there are \(n\) nodes for \(n\) keys.
      Therefore the total space is \(\Theta(n)\).
    \item \code{fresh-assoc} uses \code{empty-tree} which produces \code{empty},
      which is trivially valid.
    \item \code{assign} uses \code{tree-insert}.  The \code{tree-insert}
      function recurs left or right depending on the given key, which preserves
      the order invariants on the tree.  It builds nodes with
      \code{balanced-node} and \code{almost-balanced-node}, which automatically
      construct the height field correctly.  Insertion can increase the height
      of a tree by at most one; \code{almost-balanced-node} preserves the
      balance invariants for trees that are off-balance by at most one by
      performing one or two tree rotations as needed.
      Therefore \code{tree-insert} produces a valid \ty{Tree} at every step.
    \item \code{unassign} uses \code{tree-delete-range} for a singleton range.
      The function \code{tree-delete-range} uses \code{tree-append},
      \code{tree-filter<}, and \code{tree-filter>}.  These functions all
      preserve the key order from the original tree.  They all ultimately use
      \code{unbalanced-node}, \code{almost-balanced-node}, or
      \code{balanced-node} to construct nodes, which all maintain the height
      invariants.  The \code{unbalanced-node} function recurs on the left or
      right to ``push'' the smaller subtree down inside the larger one to find
      siblings of similar size, then calls \code{almost-balanced-node} when it
      gets there.  The result is always a balanced tree.
    \end{enumerate}
  \item
    \begin{enumerate}
    \item We use worst-case analysis for our associative mappings.
    \item \code{fresh-assoc} produces \code{empty} immediately.
    \item \code{assign} calls \code{tree-insert}, which recurs \(O(\log n)\)
      times and only calls \(\Theta(1)\)-time functions.
    \item \code{unassign} calls \code{tree-append}, \code{tree-filter>}, and
      \code{tree-filter>}.  These functions all recur at most \(O(\log n)\)
      times.  They all also call \code{unbalanced-node}.  The running time of
      \code{unbalanced-node} is odd; it runs in \(O(|\log m - \log n|)\) time
      for trees of size \(m\) and \(n\).  Its running time is proportional to
      the \emph{difference} between the heights of its arguments.  As
      \code{tree-append}, \code{tree-filter<}, and \code{tree-filter>} recur,
      their calls to \code{unbalanced-node} ``walk'' down the height of a tree.
      If one call to \code{unbalanced-node} takes several steps by going from
      height \(x+k\) to height \(x\), the next call will start at height \(x\).
      In other words, the \emph{total} time spend in calls to
      \code{unbalanced-node} is \(O(\log n)\) for any of \code{tree-append},
      \code{tree-filter<}, and \code{tree-filter>}.  Therefore all three
      functions take \(O(\log n)\) time.
    \item \code{lookup} calls \code{tree-search} which recurs at \(O(\log n)\)
      times and calls no helpers.
    \end{enumerate}
  \end{enumerate}

\item
  \begin{enumerate}
  \item A \ty{Set} is a \ty{Tree} as defined above, in which every key
    represents an element of the set and the values associated with keys are
    irrelevant.
  \item
    \begin{enumerate}
    \item Trees take \(\Theta(n)\) space as argued above.
    \item \code{empty-set} calls \code{empty-tree}, which produces a valid tree
      as argued above.
    \item \code{extend} calls \code{tree-insert}, which produces a valid tree as
      argued above.
    \item \code{without} calls \code{tree-delete-range}, which produces a valid
      tree as argued above.
    \end{enumerate}
  \item
    \begin{enumerate}
    \item Our bounds are all worst-case.
    \item \code{empty-set} takes \(O(1)\) time trivially.
    \item \code{in\?} calls \code{tree-search}, which takes \(O(\log n)\) time
      as argued above.
    \item \code{extend} calls \code{tree-insert}, which takes \(O(\log n)\) time
      as argued above.
    \item \code{without} calls \code{tree-delete-range}, which takes
      \(O(\log n)\) time as argued above.
    \end{enumerate}
  \end{enumerate}

\item
  \begin{enumerate}
  \item
\begin{verbatim}
;; A Queue is (queue Chain Chain).
(struct queue [leftmost rightmost] #:mutable #:transparent)
;; Either leftmost and rightmost are both empty, or they are both links.
;; If both are links, then they are part of a single chain of links:
;; * leftmost.left = empty
;; * rightmost.right = empty
;; * for any link L other than leftmost, L.left.right = L
;; * for any link L other than rightmost, L.right.left = L
;; * from any link L, leftmost is reachable by going left 0 or more times
;; * from any link L, rightmost is reachable by going right 0 or more times
;; These invariants must hold before and after any Queue operations.
;; Note that during Queue operations, in between mutations,
;; some or all of these invariants may be violated.

;; A Chain is either empty or (link Link String Link).
(struct link [left elem right] #:mutable #:transparent)
\end{verbatim}

  \item
    \begin{enumerate}
    \item A queue contains one link per value, and every link takes
      \(\Theta(1)\) space.  The total is therefore \(\Theta(n)\) space.
    \item \code{new-queue} produces a queue that is trivially valid.
    \item \code{add-leftmost} maintains the left and right link invariants when
      adding links, and therefore produces a valid queue.
    \item \code{drop-leftmost} maintains the left and right link invariants when
      removing links and correctly detects when the queue becomes empty; it
      therefore produces a valid queue.
    \item \code{add-rightmost} maintains the left and right link invariants when
      adding links, and therefore produces a valid queue.
    \item \code{drop-rightmost} maintains the left and right link invariants
      when removing links and correctly detects when the queue becomes empty; it
      therefore produces a valid queue.
    \item \code{concatenate} maintains the left and right link invariants, and
      ensures that every link is part of only one queue; it therefore produces a
      valid queue.
    \end{enumerate}
  \item The analysis for queues is worst-case, and by inspection every operation
    takes \(\Theta(1)\) time because there is no recursion or iteration.
  \end{enumerate}

\end{enumerate}

\end{document}

\documentclass{article}

\title{Homework 1}
\author{Carl Eastlund}
\date{Due \textbf{Wed., Jan. 30} at \textbf{9:00pm}.}

%% Fonts
\usepackage{newcent}
\usepackage{fouriernc}
\renewcommand{\ttdefault}{cmtt}

%% Margins
\usepackage[top=1in,bottom=1in,left=1in,right=1in]{geometry}

%% URL formatting and PDF hyperlinks
\usepackage{url}
\usepackage{hyperref}

\begin{document}

\maketitle

\newcommand\link[2][http://]{\href{#1#2}{\nolinkurl{#2}}}
\newcommand\http[1]{\link[http://]{#1}}
\newcommand\https[1]{\link[https://]{#1}}
\newcommand\email[1]{\link[mailto:]{#1}}

\begin{quotation}

  \noindent \textbf{Collaboration Policy:} Your work on this assignment must be
  your own.  You \emph{may not} copy files from other students in this class,
  from people outside of the class, from the internet, or from any other source.
  You \emph{may not} share files with other students in this class.

  \medskip

  \noindent You \emph{may} discuss the problems, concepts, and general
  techniques used in this assignment with other students, so long as you do not
  share actual solutions.

  \medskip

  \noindent If you are in doubt about what you \emph{may} and \emph{may not} do,
  ask the course instructor before proceeding.  If you violate the collaboration
  policy, you will receive a zero as your grade for this entire assignment and
  you will be reported to OSCCR (\link{northeastern.edu/osccr}).

\end{quotation}

\begin{enumerate}

  \pagebreak[1]
\item
  A \emph{selection} algorithm chooses the \(k\)th-from-smallest element in a
  list.  For instance, \(\mathsf{select}(0,[20,10,50])=10\) because 10 is the
  smallest element; \(\mathsf{select}(1,[20,10,50])=20\) because 20 is one from
  the smallest element; and \(\mathsf{select}(2,[20,10,50])=40\) because 50 is
  two from the smallest element.  The \emph{quickselect} algorithm performs
  selection using a method similar to quicksort.
  \begin{enumerate}
  \item
    Implement quickselect in Racket.  Given a list \(\ell\) of length \(n\) and
    an index \(k\) such that \(0 \leq k < n\), quickselect performs the
    following steps:
    \begin{enumerate}
    \item
      First, select a pivot element \(p\); for our purposes, use the first
      element of the list.
    \item
      Second, split the list into two partitions: \(\ell_1\), containing all
      elements less than \(p\), and \(\ell_2\), containing all elements greater
      than \(p\).
    \item
      Third, compute the lengths \(n_1\) and \(n_2\) of \(\ell_1\) and
      \(\ell_2\), respectively.
    \item
      Fourth, if \(k < n_1\), then the \(k\)th-from-smallest element of \(\ell\)
      is the \(k\)th-from-smallest element of \(\ell_1\).  Recursively compute
      and return that value.
    \item
      Fifth, if \(k \geq n - n_2\), then the \(k\)th-from-smallest element of
      \(\ell\) is the \((k - [n - n_2])\)th-from-smallest element of \(\ell_2\).
      Recursively compute and return that value.
    \item
      Otherwise, the \(k\)th-from-smallest element of \(\ell\) is \(p\) itself.
      Return \(p\).
    \end{enumerate}
  \item
    Like quicksort, quickselect has worst-case performance when the pivot is
    consistently chosen poorly: one partition has \(n-1\) elements, and the
    desired index is in that partition.
    \begin{enumerate}
    \item State the worst-case running time of quickselect as a recurrence.
    \item Solve the recurrence using the master method, recursion trees, or
      summations.
    \end{enumerate}
  \item
    Again like quicksort, quickselect has best-case performance when the pivot
    is consistently chosen well: both partitions have \(\lfloor{n/2}\rfloor\)
    elements.
    \begin{enumerate}
    \item State the best-case running time of quickselect as a recurrence.
    \item Solve the recurrence using the master method, recursion trees, or
      summations.
    \end{enumerate}
  \end{enumerate}

  \pagebreak[1]
\item
  The \emph{stooge sort} algorithm is named for the Three Stooges comedy bit
  where each member hits the other two.  Given a list of length three or more,
  the algorithm splits the list into thirds.  The stooge sort algorithm proceeds
  by sorting the first two thirds, then the last two thirds, and finally the
  first two thirds again.
  \begin{enumerate}
  \item Implement stooge sort in Racket.
  \item State the running time of stooge sort as a recurrence.
  \item Solve the recurrence by the master method, or demonstrate that the
    master method does not apply.
  \end{enumerate}

\end{enumerate}

\end{document}

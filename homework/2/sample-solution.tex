\documentclass{article}

\title{Homework 2---Sample Solution}
\author{Carl Eastlund}
\date{\today}

%% Fonts
\usepackage{newcent}
\usepackage{fouriernc}
\renewcommand{\ttdefault}{cmtt}

%% Margins
\usepackage[top=1in,bottom=1in,left=1in,right=1in]{geometry}

%% Math formatting
\usepackage{nicefrac}
\usepackage{amssymb}

%% URL formatting and PDF hyperlinks
\usepackage{url}
\usepackage{hyperref}

\newcommand\link[2][http://]{\href{#1#2}{\nolinkurl{#2}}}
\newcommand\http[1]{\link[http://]{#1}}
\newcommand\https[1]{\link[https://]{#1}}
\newcommand\email[1]{\link[mailto:]{#1}}

\begin{document}

\maketitle

\begin{enumerate}

  \pagebreak[2]
\item
  \begin{enumerate}
    \addtocounter{enumii}{1}
  \item
    Like quicksort, quickselect has worst-case performance when the pivot is
    consistently chosen poorly: one partition has \(n-1\) elements, and the
    desired index is in that partition.
    \begin{enumerate}
    \item State the worst-case running time of quickselect as a recurrence.
      \paragraph{Solution:} \[T(n) = T(n-1) + n\]

      %\bigskip

    \item Solve the recurrence using the master method, recursion trees, or
      summations.
      \paragraph{Solution:} Solution using summations.
      \[T(n) = \sum_{i=1}^{n} i = \frac{n^2+n}{2} \in \Theta(n^2)\]

      %\bigskip

    \end{enumerate}
  \item
    Again like quicksort, quickselect has best-case performance when the pivot
    is consistently chosen well: both partitions have at most \(\frac{n}{2}\)
    elements.
    \begin{enumerate}
    \item State the best-case running time of quickselect as a recurrence.
      \paragraph{Solution:} \[T(n) = T(\frac{n}{2}) + n\]

      %\bigskip

    \item Solve the recurrence using the master method, recursion trees, or
      summations.
      \paragraph{Solution:} Solution using the master method where
      \(a=1\), \(b=2\), and \(f(n)=n\).  Trying case 3 because
      \(n^{\log_b a} = n^0 = 1 \in o(f(n))\).  Choosing \(\frac{1}{2}\) for
      \(\epsilon\) gives us
      \(f(n) \in \Omega(n^{\log_b a + \epsilon}) = \Omega(\sqrt{n})\).  Choosing
      \(c=\frac{1}{2}\) gives us
      \(af(\frac{n}{b})=f(\frac{n}{2})=\frac{n}{2}=\frac{1}{2}n=cf(n)\).
      Therefore case 3 applies, and we have a solution for the recurrence.
      \[T(n) \in \Theta(n)\]

      %\bigskip

    \end{enumerate}
  \end{enumerate}

  \pagebreak[2]
\item
  \begin{enumerate}
    \addtocounter{enumii}{1}
  \item State the running time of stooge sort as a recurrence.
    \paragraph{Solution:} \[T(n)=3T(\frac{2}{3}n)+1\]

    \newpage
  \item Solve the recurrence using the master method, recursion trees, or
    summations.
    \paragraph{Solution:} Solution using the master method where
    \(a=3\), \(b=1.5\), and \(f(n)=1\).  Trying case 1 where
    \(\epsilon = \log_{1.5} 2\); specifically,
    \(\log_b a - \epsilon = \log_{1.5} 3 - \log_{1.5} 2 = 1\).  Therefore
    \(f(n) = 1 \in O(n^1) = O(n^{\log_b a - \epsilon})\), so case 1 applies.
    (Of course, \(\epsilon\) does not have to be chosen so cleverly to make
    \(\log_b a - \epsilon = 1\); any number between 0 and
    \(\log_{1.5} 3 \approxeq 2.71\) works.)
    \[T(n) \in \Theta(n^{\log_{1.5} 3E}) \approxeq \Theta(n^{2.71})\]
  \end{enumerate}

  \pagebreak[2]
\item \textbf{Note:} For each case, we note the simplified \(\Theta\)
  form of each \(f(n)\) first.
  \begin{enumerate}
  \item \(f(n) = 5n^{1.25} + 3n\log n + 2 n\sqrt{n}\).
    \textbf{Note:} \(f(n) \in \Theta(n\sqrt{n}) = \Theta(n^{1.5})\)
    \begin{enumerate}
    \item \(g(n) = n^2\).
      \textbf{Solution:} \(f(n) \in o(g(n))\)
    \item \(g(n) = n^{\nicefrac{3}{2}}\).
      \textbf{Solution:} \(f(n) \in \Theta(g(n))\)
    \item \(g(n) = n\log n\).
      \textbf{Solution:} \(f(n) \in \omega(g(n))\)
    \item \(g(n) = n\).
      \textbf{Solution:} \(f(n) \in \omega(g(n))\)
    \end{enumerate}
  \item \(f(n) = n (\log \frac{n}{2})^2\).
    \textbf{Note:}
    \(f(n)
    = n (\log n - \log 2)^2
    = n(\log n)^2 - 2\log2(n\log n) + (\log 2)^2n
    \in \Theta(n (\log n)^2)\)
    \begin{enumerate}
    \item \(g(n) = n\).
      \textbf{Solution:} \(f(n) \in \omega(g(n))\)
    \item \(g(n) = n \sqrt{n}\).
      \textbf{Solution:} \(f(n) \in o(g(n))\)
    \item \(g(n) = n (\log n)^2\).
      \textbf{Solution:} \(f(n) \in \Theta(g(n))\)
    \item \(g(n) = n \log n\).
      \textbf{Solution:} \(f(n) \in \omega(g(n))\)
    \end{enumerate}
  \item \(f(n) = 2^{2n}\).
    \textbf{Note:} \(f(n) = 4^n \in \Theta(4^n)\)
    \begin{enumerate}
      \item \(g(n) = n^{65536}\).
        \textbf{Solution:} \(f(n) \in \omega(g(n))\)
      \item \(g(n) = 2^n\).
        \textbf{Solution:} \(f(n) \in \omega(g(n))\)
      \item \(g(n) = 3^n\).
        \textbf{Solution:} \(f(n) \in \omega(g(n))\)
      \item \(g(n) = 4^n\).
        \textbf{Solution:} \(f(n) \in \Theta(g(n))\)
      \item \(g(n) = 5^n\).
        \textbf{Solution:} \(f(n) \in o(g(n))\)
      \item \(g(n) = n!\).
        \textbf{Solution:} \(f(n) \in o(g(n))\)
    \end{enumerate}
  \end{enumerate}

  \pagebreak[2]
\item
  \begin{enumerate}
  \item \(T(n) = 5T(\frac{1}{4}n) + (\frac{5}{4})^n\)
    \paragraph{Solution:}
    \(a=5\), \(b=4\), and \(f(n) = (\frac{5}{4})^n\).  An exponential \(f(n)\)
    suggests that case 3 applies.  Clearly
    \(f(n) \in \Omega(n^{\log_b a + \epsilon})\) regardless of what \(\epsilon\)
    we choose.  We must show that
    \(5f(\frac{n}{b}) = \leq c(\frac{5}{4})^n\) for
    sufficiently large \(n\).
    \[\begin{array}{rcll}
    5(\frac{5}{4})^{\frac{n}{4}} &\leq& c(\frac{5}{4})^n \\
    5c^{-1} &\leq& (\frac{5}{4})^{\frac{3}{4}n} \\
    10 &\leq& (\frac{5}{4})^{\frac{3}{4}n} &
    \mbox{if we choose \(c=\frac{1}{2}\).} \\
    10,000 &\leq& (\frac{5}{4})^n \\
    41.3 \approxeq \log_{\frac{5}{4}} 10,000 &\leq& n \\
    \end{array}\]
    The equation therefore holds for \(c=\frac{1}{2}\) and \(n\) over 42.
    \[T(n) \in \Theta((\frac{5}{4})^n)\]
    \newpage
  \item \(T(n) = 9T(\frac{1}{3}n) + n^2\sqrt{\log n}\)
    \paragraph{Solution:}
    \(a=9\), \(b=3\), and \(f(n) = n^2\sqrt{\log n}\).
    In this case, \(n^{\log_b a}=n^2\).  This is asymptotically smaller than
    \(f(n)\), which suggests case 3.  However, there is no \(\epsilon\) we can
    choose such that \(n^2\sqrt{\log n} \in \Omega(n^{2+\epsilon})\).
    Therefore, the master method does not apply.
    \paragraph{Note:} Exercise 4.6-2 in the text (CLRS, 3rd Edition) extends
    case 2 of the master method to allow
    \(f(n) \in \Theta(n^{\log_b a}(\log n)^k)\) for any \(k\geq 0\); the
    solution is then \(T(n) \in \Theta(n^{\log_b a}(\log n)^{k+1})\).  If we use
    that extension, we can show in this case that
    \(T(n) \in \Theta(n^2(\log n)^{1.5})\).
  \item \(T(n) = 2T(\frac{1}{4}n) + \sqrt[3]{n}\)
    \paragraph{Solution:}
    \(a=2\), \(b=4\), and \(f(n)=n^{\frac{1}{3}}\).
    In this case, \(n^{\log_b a} = n^{\frac{1}{2}}\).
    Since \(f(n)\) is slower by exactly a factor of \(n^{\frac{1}{6}}\),
    we use case 1 where \(\epsilon=\frac{1}{6}\).
    \[T(n)=\Theta(\sqrt{n})\]
  \item \(T(n) = 2T(\frac{2}{3}n) + (\log n)^2\)
    \paragraph{Solution:}
    \(a=2\), \(b=1.5\), and \(f(n)=(\log n)^2\).
    In this case, \(n^{\log_b a} = n^{\log_{1.5} 2} \approxeq n^{1.71}\).
    Case 1 applies where \(\epsilon\) is any number between 0 and
    \(\log_{1.5} 2\), exclusive.
    \[T(n) = \Theta(n^{\log_{1.5} 2}) \approxeq \Theta(n^{1.71})\]
  \end{enumerate}

\end{enumerate}

\end{document}
